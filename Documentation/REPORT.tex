\documentclass[12pt,a4paper]{report}
\usepackage[latin1]{inputenc}
\usepackage{amsmath}
\usepackage{amsfonts}
\usepackage{amssymb}
\usepackage{amsthm}
\usepackage{graphicx}
\usepackage[square,sort,comma,numbers]{natbib}
\usepackage[left=1.25in,right=1in,top=1in,bottom=1in]{geometry}
%\newtheorem{chapter}{Chapter}
\newtheorem{theorem}{Theorem}[chapter]
\newtheorem{example}{Example}[chapter]
\newtheorem{corollary}[theorem]{Corollary}
\newtheorem{lemma}[theorem]{Lemma}
\newtheorem{remark}[theorem]{Remark}
\newtheorem{definition}[theorem]{Definition}
\usepackage{makeidx}
\usepackage{enumerate}
\newtheorem{conjecture}[theorem]{Conjecture}
\newtheorem{principle}[theorem]{Principle}
\newtheorem{result}[theorem]{Result}
\usepackage{setspace}
\usepackage{hyperref}
\setstretch{1.5}
\hyphenpenalty
\exhyphenpenalty
\begin{document}
    %\baselinestretch{1.5 cm}
    
%-----Title Page-------------------------------------------------

\begin{titlepage}
    {\centering{\LARGE{\textbf{CSV to DB converter}}}\\\vspace*{\fill} {\large{A project report\\[0.2cm] submitted by}}\\\vspace*{\fill} {\Large{\itshape Mohd Zaid}}\\(Class 12)\\ \vspace*{\fill}{\large under the supervision of}\\[0.2cm]
    {\Large \textit{Mrs. Swati Saxena}}\\\vspace*{\fill}{\large in completion of project\\ submitted as yearly project} \\
    {\Large For class 12.}  \\\vspace*{\fill} to the\\\vspace*{\fill}
    
    \includegraphics[scale=0.5]{logo.png}\\[0.5cm]
    {\large Kendriya Vidyalaya \\ Andrews ganj} \\ November 2022

\newpage

    Kendriya Vidyalaya ANDREWS GANJ NEW DELHI\\
    \vspace{1cm}
    {\LARGE \textbf{DECLARATION}}\\}
    \vspace{1cm}
    I hereby declare that the work reported in the project titled \textbf{``CSV to DB converter"} submitted for the fulfillment of yearly project, is record of my work carried out under the supervision of \textit{Mrs. Swati Saxsena}.
    
    \vspace{3cm}
    \begin{flushright}
        {\Large Mohd Zaid}\\(Class 12)\\
        Kendriya Vidyalaya\\Andrews Ganj ND\\November 2022
    \end{flushright}	

\newpage
    {\centering KENDRIYA VIDYALAYA ANDREWS GANJ NEW DELHI\\ 
    \vspace{1cm}
    {\Large \textbf{CERTIFICATE}}\\}
    \vspace{.50cm}		
		
    It is certified that \textbf{Mohd Zaid} has submitted project under my supervision for fulfillment of Project on the topic \textbf{``CSV to DB converter"}. It is further certified that the above candidate has carried out the project work under my guidance during the academic session 2022-2022.
    
    \vspace{2cm}
    \begin{flushright}
        {\Large {Mrs. Swati Saxsena}}\\
        PGT Conputer Science\\Kendriya Vidyalaya\\New Delhi - 110025
    \end{flushright}
		
\newpage
    \begin{center}
        {\LARGE \textbf{ACKNOWLEDGEMENT}}\\
    \end{center}
    \vspace{1cm}
    As a matter of first importance, I offer my genuine thanks to my supervisor Mrs. Swati Saxsena, PGT CS, Kendriya Vidyalaya. I appreciate his support and help during the project work. Own Ideas/Acknowledgement\\
    \vspace{1cm}
    
    \begin{flushright}
        {\large Mohd Zaid}\\(Nov 2022**)\\
    \end{flushright}

\end{titlepage}

%---------End of Title Page-----------------------------------------------

%---------Contents Page---Don't Change-----------------------------------

\newpage

    \tableofcontents

\newpage

%----------Your Content Starts from here----------------------------------

    \chapter{Introduction}
	\vspace{1cm}
    This is report of project named \textbf{CSV to DB converter} which is to be submitted as my yearly project for class 12.
        \section{Goals}
        Through this project i will be attempting to make a program which takes a \textbf{CSV}\texttt{(Comma separated values)} file as input and copies its content to \texttt{mySql Database} format.\\
        \section{Preliminaries}
        Some basic knowledge are required to build and understand this program.
        \begin{enumerate}
            \item \textbf{mySql} to edit tables and databases.
            \item \textbf{Python} to code basic functionalities
            \item \textbf{Csv} to read data from file.csv file.
            \item \textbf{Loops} to iterate through rows.
        \end{enumerate}
 
\newpage
	
    \chapter{Header files and their uses} 
        \vspace{.6cm}
        \section{Frontend.stylers}
        It is a local package. Made with intention to provide Interface inside terminal.\\
        It is imported using command
        \begin{verbatim}
        from Frontend.Stylers import Console, SQL
        \end{verbatim}
        Both Console and SQL have calling functions which display respective informations.

        \section{Csv}
        This module is imported to read data from CSV file.
        \begin{verbatim}
        file data = csv.reader( csv file name.csv )
        \end{verbatim}
        
        \section{sqlite3}
        This module is imported to Create connection to on-disk file and add entries to it.
        \begin{verbatim}
        connection = sqlite3.connect( file name.db )
        cursor = connection.cursor()
        commands = cursor.execute( sqlite commands )
        \end{verbatim}
        
        \section{os}
        This module is imported to Check if file that has been provided by user exists or not. If not then shows error, continues otherwise.
        \begin{verbatim}
        if not os.path.exists( csv file path.csv ) : ...
        \end{verbatim}
        
        \section{mySql}
        This module is imported to Create connections to localhost database and execute commands in there.
        \begin{verbatim}
        try: import mysql except:ModuleNotFoundError:...
        \end{verbatim}
        this is imported in try and except block as it may not be present on user's system or maybe not correct version.\\
        If it is present then again
        \begin{verbatim}
        connection=mysql.connector.connect(
                                    host='localhost',
                                    passwd=dbpasswd,
                                    dbname=dbname )
                                    
        cusrsor = connection.cursor()
        \end{verbatim}
        All again in try and Except block because host may not be present or password may be wrong.
        
        
        
\newpage

    \chapter{Code}
        \vspace{3cm}
        \section{Doc-strings and file information.} 
        Explains code inside file.
        \begin{verbatim}    
    """
        This programme converts csv entries into database entries.
            -   User can choose to use server database or file `.db`.
            -   Also supports xls to db entry in future.
    """
        
    # Info
    __version__ = 1
    __author__  = 'Zaid'
        \end{verbatim}
\newpage

        \section{Imports and exception handling}
        \begin{verbatim}
    # Imports from other files.
    from Frontend.Stylers import Console, SQL
    
    # Imports from inbulit library.
    import csv
    import os
    import sqlite3
    
    # Dependent Imports
    # If mysql.conncetor (python) is not installed.
    # Download - python pip install mysql-connector-python
    try:
        import mysql.connector
    except ModuleNotFoundError as MNFErr:
        ...
        \end{verbatim}
        \section{Menu}
        \begin{verbatim}
    options_1 = [
    '\nWhere do you want to make entry:',
    '*   [f] -  To a "db" file using sqlite.',
    '*   [d] -  To a existing "db" in system.\n']
    DB_TYPE: str = Console.Input(*options_1, sep='\n')
        \end{verbatim}  
\newpage
        \section{Navigation and Inputs}
        \begin{verbatim}
    # making connection and cursor
    if 'f' in DB_TYPE.lower():
        # Gathering informations.
        csv_file_path:str = Console.Input(...)
        db_file_name:str  = Console.Input(...)
        db_table_name:str = Console.Input(...)
        # Checking informations.
        if not os.path.exists('\\Database'): os.makedirs('\\Database')
        # Processing to transfer data
        connection = sqlite3.connect(f'Database\\{db_file_name}.db')
        cursor     = connection.cursor()
    elif 'd' in DB_TYPE.lower():
        # Gathering informations
        csv_file_path:str = Console.Input(...)
        db_passwd   :str  = Console.Input(...)
        db_name     :str  = Console.Input(...)
        db_table_name:str = Console.Input(...)
        # Creating cursor
        try:
            connection = mysql.connector.connect(...)
            cusrsor = connection.cursor()
        except NameError:
            Console.Error(...)
            Console.Log('Using sqlite')
            connection = sqlite3.connect(f'Database\\{db_name}.db')
            cursor     = connection.cursor()
        except Exception as Exc:
            Console.Error(...)
            Console.Log('Using sqlite')
            connection = sqlite3.connect(f'Database\\{db_name}.db')
            cursor     = connection.cursor()\end{verbatim}  
\newpage
        \section{Commands}
        \begin{verbatim}
    # All commands
    CREATE_TABLE = '''CREATE TABLE IF NOT EXISTS {} (
                            {});'''
    
    COLUMNS = 'VARCHAR,\n\t'
    
    INSERT_INT = '''INSERT INTO {}
                            VALUES ("{}")'''
    
    VALUES = '","'
        \end{verbatim}
        \section{Entry loop}
        \begin{verbatim}
    # Convertor
    with open( csv_file_path, 'r' ) as csv_file:
        file_data = csv.reader( csv_file )
    
        for index, data in enumerate(file_data):
            try:
                if index == 0:
                    cursor.execute( CREATE_TABLE.format( ..., ... ) )
                    continue
                cursor.execute( INSERT_INT.format( ..., ... )) 
            except Exception as Exc:
                Console.Error( Exc )
        
        connection.commit()
        \end{verbatim}
\newpage

    \chapter{Uses}
        \vspace{1cm}
        This program can be used to transfer large tabular entries from csv files to school database.
        
\newpage

%--------End of Content----------------------------------------

%--------Bibliography-------------------------------------

\bibliographystyle{amsplain}

    \begin{thebibliography}{10}
		\bibitem{beer1} \href{https://www.w3schools.com/}{W3Schools} Tutorial on \href{https://www.w3schools.com/sql/}{SQL} used to learn and implement basic SQL codes.
        
		\bibitem{beer1} \href{https://www.w3schools.com/}{W3Schools} Tutorial on \href{https://www.w3schools.com/python/}{Python} used to learn and implement basic Python codes.
        
  
		\bibitem{beer1} \href{https://www.w3schools.com/}{W3Schools} Tutorial on \href{https://www.w3schools.com/python/python_file_handling.asp}{Python file handling} used to learn and implement basic file handling codes.
		
    \end{thebibliography}

%---------End of Bibliography---------------------------------

\end{document}